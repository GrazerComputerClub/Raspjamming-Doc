
\subsection{dd Kommandozeilenprogramm}

Die erhaltene Image-Datei kann mit dem Kommandozeilenprogramm dd auf eine MicroSD-Karte �bertragen werden.\\
\textbf{Es ist unbedingt vor dem Ausf�hren des Befehls zu pr�fen, ob das angegebene Laufwerk bzw. Device auch der vorgesehenen MicroSD-Karte entspricht!}\\ 
Bei USB-Kartenlesern bzw. USB-Adaptern ist die Ermittlung des Devices leicht �ber die Systemmeldungen m�glich. 

\begin{console}
	dmesg | tail -n 10
\end{console}

\begin{screensmall}
	scsi 3:0:0:0: Direct-Access     MXT-USB  Storage Device   1308 PQ: 0 ANSI: 0 CCS
	sd 3:0:0:0: Attached scsi generic sg1 type 0
	sd 3:0:0:0: [sdb] 15730688 512-byte logical blocks: (8.05 GB/7.50 GiB)
	sd 3:0:0:0: [sdb] Write Protect is off
	sd 3:0:0:0: [sdb] Mode Sense: 03 00 00 00
	sd 3:0:0:0: [sdb] No Caching mode page found
	sd 3:0:0:0: [sdb] Assuming drive cache: write through
	sdb: sdb1 sdb2
	sd 3:0:0:0: [sdb] Attached SCSI removable disk
	EXT4-fs (sdb2): mounted filesystem with ordered data mode. Opts: (null)
\end{screensmall}

Die MicroSD-Karte wurde im Beispiel als Device "`sdb"' �ber einen USB-Adapter eingebunden. Nun kann man die Image-Datei mit dem Programm dd auf die MicroSD-Karte �bertragen. Mit dem Parameter "`of"' muss der komplette Device-Name, in diesem Fall "`/dev/sdb"', angegeben werden. Bei Parameter "`if"' wird die entpackte Image-Datei angeben. Die Bl�ckgr��e bzw. der Cache wird mit Parameter "`bs"' gesetzt. Eine gr��ere Blockgr��e erh�ht die Schreibgeschwindigkeit. Sie wird im Beispiel mit 4~MB angegeben.\\
Zu Beachten ist, dass der USB-Massenspeicher m�glicherweise bereits automatisch gemountet wurde. Dann sollte man die Partitionen mit dem Befehl "`umount"' zuerst auswerfen. 


\begin{console}
	umount /dev/sdb1 /dev/sdb2
	dd if=2019-04-26-Raspjamming-full.img of=/dev/sdc bs=4M
\end{console}

\begin{screensmall}
	2623+0 Datens�tze ein
	2623+0 Datens�tze aus
	2387266048 Bytes (2,4 GB) kopiert, 247,6147 s, 10,4 MB/s
\end{screensmall}
