\subsection{wiringPi mit C\# (Mono)}

Um mit C\# Zugriff auf Funktionen der GPIOs zu erhalten, wird die wiringPi C-Bibliothek (siehe \ref{wiringPi}) und Mono ben�tigt.\\
Mono ist die quelloffene Implementierung von Microsofts .NET Framework und wird unter der MIT Lizenz angeboten.\\

Damit die C\# Projekte aus Kapitel \ref{Projekte} kompiliert werden k�nnen, ben�tigt man einen Wrapper f�r die wiringPi Funktionen. Nachfolgend ist ein Auszug einer Implementierung eines C\# wiringPi Wrappers angegeben.\\

\lstset{language=C, caption=, label=WiringPiCS, frame=single, basicstyle=\ttfamily
	\footnotesize, breakatwhitespace=false, showstringspaces=false, showtabs=false, tabsize=2 }
\lstinputlisting{source/WiringPi.cs}

Ein vollst�ndiger Wrapper kann z.~B. unter \url{https://github.com/EvilVir/WiringPi.NET/raw/master/Wrapper/WiringPi.cs} bzw. \url{https://goo.gl/isrNeJ} heruntergeladen werden.\\
Am vorbereitet Image ist der Wrapper unter \texttt{/home/pi/Projekte/wiringPi.cs} zu finden.\\
