
%\textbf{Anschluss:} 

\begin{figure}[ht]
  \centering
  \includegraphics[scale=0.25]{images/HC-SR04_Steckplatine.png}	
  %	\caption{}
  \label{DHT22_Steckplatine}
\end{figure}

%\textbf{Schaltplan:} 

\begin{figure}[ht]
	\centering
	\includegraphics[scale=0.25]{images/HC-SR04_Schaltplan.png}	
	%	\caption{}
	\label{DHT22_Steckplatine}
\end{figure}

\ExerciseBox{
Miss die Distanz [Beispiele]\\
Gib den Wert zyklisch am TM1637 Display aus [Beispiel Python]\\
Filtere die gemessene Distanz\\
Ermittle die Geschwindigkeit eines bewegten Objektes\\
Verwende die korrigierte Schallgeschwindingkeit bei aktueller Lufttemperatur vom DHT22}


\textbf{C:} 

\begin{console}
	git clone https://github.com/mstroh76/Sensors-WiringPi.git
	cd Sensors-WiringPi/HC-SR04	
	g++ -o HC-SR04 *.cpp -lwiringPi	
	sudo ./HC-SR04
\end{console}

\textbf{C\#:}

\begin{console}
	git clone https://github.com/chirndler/wiringpi.net.sensors.git
	cd wiringpi.net.sensors
	xbuild /p:Configuration=Release wiringpi.net.sensors.sln
	cd bin/Release/
	sudo mono wiringpi.net.sensors.sample.exe 2
\end{console}

\textbf{Python:}

\lstset{language=Python, caption=, 
        label=HCSR04Program, frame=single, basicstyle=\ttfamily
	      \footnotesize, breakatwhitespace=false, showstringspaces=false, 
        showtabs=false, tabsize=2 }
\lstinputlisting{source/HC_SR04.py}

