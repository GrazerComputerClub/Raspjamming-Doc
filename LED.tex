
\begin{figure}[ht]
  \centering
  \includegraphics[scale=0.25, angle=-90]{images/LED_Steckplatine.png}	
  %	\caption{}
  \label{LED_Steckplatine}
\end{figure}

\begin{figure}[ht]
	\centering
	\includegraphics[scale=0.25]{images/LED_Schaltplan.png}	
	%	\caption{}
	\label{LED_Schaltplan}
\end{figure}

\subsection{Shell}

\begin{console}
	gpio -g mode 17 out
	gpio -g write 17 1
	gpio -g write 17 0
	gpio -g mode 17 in
\end{console}

%https://de.scribd.com/doc/101830961/GPIO-Pads-Control2

% gpio drive 0 0  2 mA
% gpio drive 0 3  8 mA <- default
% gpio drive 0 7  16 mA


\subsection{Python}

\subsection{C\#}

\subsection{C}

\begin{console}
	geany &
\end{console}

Nun kann man ein neues Projekt erstellen, dazu w�hlt man \texttt{Projekt} $\rightarrow$ \texttt{Neu...}. Dann Gibt man den Namen des Projekts an. Das Anlegen der Verzeichnisse muss man auch noch best�tigen. 

\begin{figure}[ht]
	\centering
	\includegraphics[scale=0.48]{images/Geany_Projekt.png}
	\includegraphics[scale=0.42]{images/Geany_Projekt2.png}
%	\caption{}
	\label{Geany-create}
\end{figure}

Danach w�hlt man \texttt{Datei} $\rightarrow$ \texttt{Speichern unter} um die unbenannte Datei mit dem Namen "`Blink.c"' speichern zu k�nnen. Nun kann man den folgenden C-Source eingeben.

\lstset{language=C, caption=, label=LEDProgram, frame=single, basicstyle=\ttfamily
	\footnotesize, breakatwhitespace=false, showstringspaces=false, showtabs=false, tabsize=2 }
\lstinputlisting{source/Blink.c}

Jetzt darf man nicht vergessen im Men� unter \texttt{Erstellen} $\rightarrow$ \texttt{Kommandos zum Erstellen konfigurieren} die Wiring Pi Library mit "`-lwiringPi"' bei Compile und Build zu erg�nzen.

\begin{figure}[ht]
	\centering
	\includegraphics[scale=0.48]{images/Geany_Create_Wiringpi.png}
	%	\caption{}
	\label{Geany-create}
\end{figure}

Nun kann man das Projekt mit den Ziegel-Icon \includegraphics[scale=0.4]{images/Geany_Icon_Erstellen.png} erstellen bzw. kompilieren und danach mit dem Zahnrad-Icon \includegraphics[scale=0.4]{images/Geany_Icon_Ausfuehren.png}  ausf�hren.  


%\lstset{language=C, caption='LED.c' C Source Code, label=LEDProgram, frame=single, basicstyle=\ttfamily
%	\footnotesize, breakatwhitespace=false, showstringspaces=false, showtabs=false, tabsize=2 }
%\lstinputlisting{LED.c}

%Download: \urlsmall{https://github.com/mstroh76/RaspberryPiProjekte/raw/master/Elektronik/LED.c}

%
%\begin{console}
%   wget https://github.com/mstroh76/RaspberryPiProjekte/raw/master/Elektronik/test_output.c
%	gcc -Wall -o LED LED.c -lwiringPi
%	./LED
%\end{console}
%
%\begin{screensmall}
%	Raspberry Pi wiringPi LED program using GPIO17 (press ctrl+c to quit)
%	LED blinking...
%	on off on off on off on off on off on off on off on^C off
%	LED off
%\end{screensmall}
