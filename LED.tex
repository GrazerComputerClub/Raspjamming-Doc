
\begin{figure}[ht]
  \centering
  \includegraphics[scale=0.25]{images/LED_Steckplatine.png}	
  \includegraphics[scale=0.25]{images/LED_Schaltplan.png}
  %	\caption{}
  \label{LED_Steckplatine}
\end{figure}


\ExerciseBox{
Lasse die LED blinken [Beispiele]\\
Ver�ndere die Helligkeit der LED in dem du ein PWM-Signal erzeugst}


\subsection{Shell}

\begin{console}
	gpio -g mode 17 out
	gpio -g write 17 1
	gpio -g write 17 0
	for i in {1..3}; do gpio -g write 17 1; sleep 1; gpio -g write 17 0; sleep 1; done
	gpio -g mode 17 in
\end{console}

%https://de.scribd.com/doc/101830961/GPIO-Pads-Control2

% gpio drive 0 0  2 mA
% gpio drive 0 3  8 mA <- default
% gpio drive 0 7  16 mA

\clearpage
\subsection{C}

\begin{console}
	geany &
\end{console}

Zuerst wird ein neues Projekt erstellt. Dazu w�hlt man im Men� \texttt{Projekt} $\rightarrow$ \texttt{Neu...}. Dann gibt man den Namen des Projekts an, z.~B. Blink. Das Anlegen der Verzeichnisse muss danach auch noch best�tigt werden.

\begin{figure}[ht]
	\centering
	\includegraphics[scale=0.48]{images/Geany_Projekt.png}
	\includegraphics[scale=0.42]{images/Geany_Projekt2.png}
	%	\caption{}
	\label{Geany-create}
\end{figure}


Weitere Einstellungen wie Zeichen f�r Einr�ckungen und Zeilenumbruch k�nnen unter \texttt{Projekt} $\rightarrow$ \texttt{Eigenschaften} vorgenommen werden.\\
Danach w�hlt man \texttt{Datei} $\rightarrow$ \texttt{Speichern unter} um die unbenannte Datei mit dem Namen "`Blink.c"' speichern zu k�nnen. Nun kann man den folgenden C-Source eingeben.

\lstset{language=C, caption=, label=LEDProgram, frame=single, basicstyle=\ttfamily
	\footnotesize, breakatwhitespace=false, showstringspaces=false, showtabs=false, tabsize=2 }
\lstinputlisting{source/Blink.c}

Im Men� muss man nun unter \texttt{Erstellen} $\rightarrow$ \texttt{Kommandos zum Erstellen konfigurieren} die WiringPi Library mit "`\textit{-lwiringPi}"' bei \texttt{Compile} und \texttt{Build} erg�nzen.

\begin{figure}[ht]
	\centering
	\includegraphics[scale=0.48]{images/Geany_Create_Wiringpi.png}
	%	\caption{}
	\label{Geany-create}
\end{figure}

Dann kann man das Projekt mit den Ziegel-Icon \includegraphics[scale=0.4]{images/Geany_Icon_Erstellen.png} erstellen bzw. kompilieren und danach mit dem Zahnrad-Icon \includegraphics[scale=0.4]{images/Geany_Icon_Ausfuehren.png} ausf�hren. Beendet wird das Programm mit der Tastenkombination \framebox{Strg}+\framebox{C}. 


\clearpage
\subsection{C\#}

\begin{console}
	geany &
\end{console}

Zuerst wird ein neues Projekt erstellt. Dazu w�hlt man im Men� \texttt{Projekt} $\rightarrow$ \texttt{Neu...}. Dann gibt man den Namen des Projekts an, z.~B. csBlink. Das Anlegen der Verzeichnisse muss danach auch noch best�tigt werden. Danach w�hlt man \texttt{Datei} $\rightarrow$ \texttt{Speichern unter} um die unbenannte Datei mit dem Namen "`\textit{Blink.cs}"' speichern zu k�nnen. Nun kann man den folgenden C\#-Source eingeben.\\

\lstset{language=C, caption=, label=LEDProgramCS, frame=single, basicstyle=\ttfamily
	\footnotesize, breakatwhitespace=false, showstringspaces=false, showtabs=false, tabsize=2 }
\lstinputlisting{source/Blink.cs}

Um das Programm kompilieren zu k�nnen, muss im Men� unter \texttt{Erstellen}
$\rightarrow$ \texttt{Kommandos zum Erstellen konfigurieren} der Pfad zur Sourcedatei des C\# WiringPi 
Wrapper (siehe \ref{WiringPiCS}) hinzugef�gt werden. Hierf�r im Textfeld \texttt{Kompilieren} den Pfad zu WiringPi.cs erg�nzen 
"`\textit{mcs /t:winexe \textquotedblleft\%f\textquotedblright WiringPi.cs /r:System,System.Drawing}"'.\\


\begin{figure}[ht]
	\centering
	\includegraphics[scale=0.48]{images/Geany_Set_cs.png}
	%	\caption{}
	\label{Geany-setpy3}
\end{figure}

Es muss die WiringPi Wrapper Datei in das Projektverzeichnis kopiert werden. 

\begin{console}
	cp ~/Projekte/WiringPi.cs ~/Projekte/csBlink/
\end{console}

Anschlie�end kann man das Programm mit dem Kompilieren-Icon \includegraphics[scale=0.4]{images/Geany_Icon_Kompilieren.png} erstellen bzw. kompilieren und mit dem Zahnrad-Icon \includegraphics[scale=0.4]{images/Geany_Icon_Ausfuehren.png} ausf�hren.
Das Programm kann mit der Tastenkombination \framebox{Strg}+\framebox{C} vorzeitig beendet werden.


\clearpage
\subsection{Python}

\begin{console}
	geany &
\end{console}

Zuerst wird ein neues Projekt erstellt. Dazu w�hlt man im Men� \texttt{Projekt} $\rightarrow$ \texttt{Neu...}. Dann gibt man den Namen des Projekts an, z.~B. PyBlink. Das Anlegen der Verzeichnisse muss danach auch noch best�tigt werden. Danach w�hlt man \texttt{Datei} $\rightarrow$ \texttt{Speichern unter} um die unbenannte Datei mit dem Namen "`\textit{Blink.py}"' speichern zu k�nnen. Nun kann man den folgenden Python-Source eingeben.

\lstset{language=Python, caption=, 
        label=LEDProgram, frame=single, basicstyle=\ttfamily
	      \footnotesize, breakatwhitespace=false, showstringspaces=false, 
        showtabs=false, tabsize=2 }
\lstinputlisting{source/Blink.py}

Um das Programm auszuf�hren zu k�nnen, muss im Men� unter \texttt{Erstellen}
$\rightarrow$ \texttt{Kommandos zum Erstellen konfigurieren} der Python 
Interpreter von Version 2 auf Version 3 umgestellt werden. Hierf�r einfach im
Textfeld \texttt{Compile} und im Textfeld \texttt{Execute} "`\textit{python}"' durch "`\textit{python3}"'
ersetzen.

\begin{figure}[ht]
	\centering
	\includegraphics[scale=0.48]{images/Geany_SetPy3.png}
	%	\caption{}
	\label{Geany-setpy3}
\end{figure}

Anschlie�end kann man das Programm mit dem Zahnrad-Icon \includegraphics[scale=0.4]{images/Geany_Icon_Ausfuehren.png} ausf�hren. Achtung, nach dem Start braucht die Initialisierung der GPIOZero Library ein paar Sekunden, bevor das Programm startet. Beendet wird das Programm mit der Tastenkombination \framebox{Strg}+\framebox{C}.


\subsection{Assembler}

Echte Hardcore-Programmierer k�nnen nat�rlich auch das Beispiel in Assembler schreiben. \texttt{;-)}

\lstset{language=[x86masm]Assembler, caption=, 
        label=LEDProgram, frame=single, basicstyle=\ttfamily
	      \footnotesize, breakatwhitespace=false, showstringspaces=false, 
        showtabs=false, tabsize=2 }
\lstinputlisting{source/Blink.asm}

Um das Programm zu kompilieren und auszuf�hren m�ssen folgende Befehle in die Kommandozeile eingegeben werden:
\begin{console}
as -o Blink.o Blink.asm 
gcc -o Blink Blink.o -lwiringPi
./Blink 
\end{console}

