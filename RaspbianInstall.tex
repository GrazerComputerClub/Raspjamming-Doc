
Raspbian ist das offizielle Betriebssystem f�r den Raspberry~Pi. F�r die Verwendung bei einem Raspberry Pi Jam wurde ein eigenes Image vom Grazer Computer Club (GC2) mit dem Namen Raspjamming erstellt. Ausgehend vom 
Raspbian Lite Image wurde noch verschiedene Entwickerlerprogramme und Werkzeuge installiert. Zus�tzlich wurde das System so eingerichtet, dass der Raspberry Pi Zero direkt �ber den USB-OTG Anschluss mit dem Host-PC verbunden werden kann. Der SSH-Dienst und WLAN wurde aktiviert. Alle L�ndereinstellung wurden von UK auf �sterreich/German ge�ndert. Ein Web-Server stellt Programme und Hilfe zur Verf�gung. Somit ist ein Betrieb auch ohne Internetverbindung m�glich. Ein Download-Link zum aktuellen Raspjamming-Image wird auf der GitHub Projekt-Seite \urlsmall{https://github.com/GrazerComputerClub/Raspjamming-Image} zur Verf�gung gestellt.\\  
F�r die Installation ben�tigt man eine mindestens 4~GB gro�e MicroSD-Karte und einen Computer mit MicroSD-Kartenleseger�t (z. B. USB-Adapter oder Karten-Slot mit Adapter).


\subsection{Linux - Etcher}

Nach dem Download, muss man die erhaltene Datei "`2019-04-26-Raspjamming-full.img.7z"' entpacken.  Dies kann �ber den Browser und Dateimanager oder �ber die Konsole erfolgen.

\begin{console}
	wget --trust-server-names http://strohmayers.com/image/2019-04-26-Raspjamming-full.img.7z
	7z e 2019-04-26-Raspjamming-full.img.7z
	rm 2019-04-26-Raspjamming-full.img.7z
\end{console}

Das grafische Programm Etcher (Download auf \url{https://www.balena.io/etcher/}) kann zum �bertragen der Image-Datei verwendet werden. Es ist vor Allem f�r Anf�nger zu empfehlen, da beim Konsolenprogramm dd das Risiko besteht, dass Daten einer falschen Partition bzw. eines Laufwerks zerst�rt werden. Das Programm muss allerdings manuell installiert werden.     

\begin{console}
	wget https://github.com/balena-io/etcher/releases/download/v1.5.43/\
	balena-etcher-electron-1.5.43-linux-x64.zip
	unzip balena-etcher-electron-1.5.43-linux-x64.zip
	sudo mv balenaEtcher-1.5.43-x64.AppImage /usr/local/bin/etcher
	etcher &
\end{console}


%Nach dem Starten wird danach gefragt ob eine Verkn�pfung zum Programm erstellt werden soll. Dies sollte man mit "`Yes"' beantworten.\\ 
Mit der Schaltfl�che "`Select image"' kann man die Image-Datei "`2019-04-26-Raspjamming-full.img"' ausw�hlen. Ist nur ein m�gliches Ziel vorhanden, wird es bereits vorausgew�hlt, z.~B. die MicroSD-Karte im Karten-Slot (/dev/memcblk0) oder im USB-Adapter (/dev/sdb). Sind mehrere m�gliche Ziele vorhanden, wird die "`Select drive"' Schaltfl�che freigeschaltet. Dann kann ein Laufwerk manuell ausgew�hlt werden.

\begin{figure}[ht]
	\centering
	\includegraphics[scale=0.3]{images/Etcher.png}
	\includegraphics[scale=0.3]{images/Etcher_2.png}
	\label{Etcher}
\end{figure}


Wenn man noch etwas �ndern will, kann die entsprechende "`Change"' Schaltfl�che ausgew�hlt werden. Zum Schluss wird der Schreibvorgang mit der "`Flash!"' Schaltfl�che gestartet. M�glicherweise wird vom Programm allerdings noch das System-Passwort abgefragt.\\ 
Das Laufwerk bzw. die Partitionen werden nun aus dem System ausgeh�ngt und der Schreibvorgang gestartet. Der Fortschritt, die durchschnittliche �bertragungsrate und die Restlaufzeit werden w�hrend des Vorgangs angezeigt.

%
\subsection{dd Kommandozeilenprogramm}

Die erhaltene Image-Datei kann mit dem Kommandozeilenprogramm dd auf eine MicroSD-Karte �bertragen werden.\\
\textbf{Es ist unbedingt vor dem Ausf�hren des Befehls zu pr�fen, ob das angegebene Laufwerk bzw. Device auch der vorgesehenen MicroSD-Karte entspricht!}\\ 
Bei USB-Kartenlesern bzw. USB-Adaptern ist die Ermittlung des Devices leicht �ber die Systemmeldungen m�glich. 

\begin{console}
	dmesg | tail -n 10
\end{console}

\begin{screensmall}
	scsi 3:0:0:0: Direct-Access     MXT-USB  Storage Device   1308 PQ: 0 ANSI: 0 CCS
	sd 3:0:0:0: Attached scsi generic sg1 type 0
	sd 3:0:0:0: [sdb] 15730688 512-byte logical blocks: (8.05 GB/7.50 GiB)
	sd 3:0:0:0: [sdb] Write Protect is off
	sd 3:0:0:0: [sdb] Mode Sense: 03 00 00 00
	sd 3:0:0:0: [sdb] No Caching mode page found
	sd 3:0:0:0: [sdb] Assuming drive cache: write through
	sdb: sdb1 sdb2
	sd 3:0:0:0: [sdb] Attached SCSI removable disk
	EXT4-fs (sdb2): mounted filesystem with ordered data mode. Opts: (null)
\end{screensmall}

Die MicroSD-Karte wurde im Beispiel als Device "`sdb"' �ber einen USB-Adapter eingebunden. Nun kann man die Image-Datei mit dem Programm dd auf die MicroSD-Karte �bertragen. Mit dem Parameter "`of"' muss der komplette Device-Name, in diesem Fall "`/dev/sdb"', angegeben werden. Bei Parameter "`if"' wird die entpackte Image-Datei angeben. Die Bl�ckgr��e bzw. der Cache wird mit Parameter "`bs"' gesetzt. Eine gr��ere Blockgr��e erh�ht die Schreibgeschwindigkeit. Sie wird im Beispiel mit 4~MB angegeben.\\
Zu Beachten ist, dass der USB-Massenspeicher m�glicherweise bereits automatisch gemountet wurde. Dann sollte man die Partitionen mit dem Befehl "`umount"' zuerst auswerfen. 


\begin{console}
	umount /dev/sdb1 /dev/sdb2
	dd if=2019-04-26-Raspjamming-full.img of=/dev/sdc bs=4M
\end{console}

\begin{screensmall}
	2623+0 Datens�tze ein
	2623+0 Datens�tze aus
	2387266048 Bytes (2,4 GB) kopiert, 247,6147 s, 10,4 MB/s
\end{screensmall}



\subsection{Windows - Rufus}

Nach dem Download, muss man die erhaltene Datei "`2019-04-26-Raspjamming-full.img.7z"' entpacken. Dazu kann das Programm 7-Zip verwendet werden (Download auf \url{http://www.7-zip.de/download.html}). Die Image-Datei kann mit dem Programm "`Rufus"' (Download auf \url{https://rufus.akeo.ie/}) auf eine MicroSD-Karte �bertragen werden. Dazu klickt man auf "`AUSWAHL"' und w�hlt dann die Datei "`2019-04-26-Raspjamming-full.img"' aus. Unter Laufwerk w�hlt man das Laufwerk aus, in dem sich die zu schreibende MicroSD-Karte befindet. Achtung: Daten von dem Laufwerk werden �berschrieben! Nach dem Dr�cken von "`Start"' beginnt der Schreibvorgang.

\begin{figure}[ht]
	\centering
	\includegraphics[scale=0.5]{images/Rufus_Raspjamming.png}
	\label{Rufus}
\end{figure}

Nun kann die MicroSD-Karte in den Raspberry Pi gesteckt und an die Versorgung angeschlossen werden.

